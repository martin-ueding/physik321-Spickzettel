\input{header.tex}

\hypersetup{
	pdftitle=
}

\ihead{physik321 Spickzettel}
\ifoot{Martin Ueding}
\ofoot{Dieses Werk bzw. Inhalt steht unter einer \href{https://creativecommons.org/licenses/by/3.0/}{CC-BY 3.0} Lizenz}

\title{physik321 Spickzettel}
\author{
	Martin Ueding \\
	\small{\href{mailto:mu@uni-bonn.de}{mu@uni-bonn.de}}
}

\begin{document}


elektrostatisches Potential:
\[
	\varphi\del{\vec x} = \frac{1}{4 \pi \varepsilon_0} \int \dif{^3 x'} \, \frac{\rho\del{\vec x'}}{\abs{\vec x - \vec x'}}
\]

magnetostatisches Potential:
\[
	\vec A = \frac{\mu_0}{4 \pi} \int \dif{^3 x'} \, \frac{\vec j\del{\vec x'}}{\abs{\vec x - \vec x'}}
\]

elektrische Feldstärke:
\[
	\vec E = - \vnabla \varphi
\]

magnetische Induktion:
\[
	\vec B = \curl \vec A
\]

Kraft auf Stromfaden:
\[
	\vec F_{1,2}
	= I_1 \oint \dif{\vec x_1} \, \times \vec B_2\del{\vec x_1}
	= \int \dif{^3 x} \, \del{\vec j \times \vec B}
\]

Drehmoment $\vec M = \vec x \times \vec F$ auf Stromdichte:
\[
	\vec M
	= \int \dif{^3 x} \, \del{\vec x \times \del{\vec j \times \vec B}}
\]

Ampère'sches Gesetz:
\[
	\vec F_{1,2}
	= \frac{\mu_0}{4 \pi} I_1 I_2 \oint \oint \frac{\dif \vec x_1 \times \del{\dif \vec x_2 \times \vec x_{1,2}}}{x_{1,2}^3}
	= - \frac{\mu_0}{4 \pi} I_1 I_2 \oint \oint \inner{\dif x_1}{\dif x_2} \frac{\vec x_{1,2}}{x_{1,2}^3}
\]

%\IfFileExists{\bibliographyfile}{
%	\bibliography{\bibliographyfile}
%	\bibliographystyle{plain}
%}{}

\end{document}

% vim: spell spelllang=de
