% Copyright © 2012-2013 Martin Ueding <dev@martin-ueding.de>
%
\documentclass[11pt, ngerman, fleqn]{scrartcl}

\usepackage{graphicx}

%%%%%%%%%%%%%%%%%%%%%%%%%%%%%%%%%%%%%%%%%%%%%%%%%%%%%%%%%%%%%%%%%%%%%%%%%%%%%%%
%                                Locale, date                                 %
%%%%%%%%%%%%%%%%%%%%%%%%%%%%%%%%%%%%%%%%%%%%%%%%%%%%%%%%%%%%%%%%%%%%%%%%%%%%%%%

\usepackage{babel}
\usepackage[iso]{isodate}

%%%%%%%%%%%%%%%%%%%%%%%%%%%%%%%%%%%%%%%%%%%%%%%%%%%%%%%%%%%%%%%%%%%%%%%%%%%%%%%
%                          Margins and other spacing                          %
%%%%%%%%%%%%%%%%%%%%%%%%%%%%%%%%%%%%%%%%%%%%%%%%%%%%%%%%%%%%%%%%%%%%%%%%%%%%%%%

\usepackage[activate]{pdfcprot}
\usepackage[left=3cm, right=2cm, top=2cm, bottom=2cm]{geometry}
\usepackage[parfill]{parskip}
\usepackage{setspace}

\setlength{\columnsep}{2cm}

%%%%%%%%%%%%%%%%%%%%%%%%%%%%%%%%%%%%%%%%%%%%%%%%%%%%%%%%%%%%%%%%%%%%%%%%%%%%%%%
%                                    Color                                    %
%%%%%%%%%%%%%%%%%%%%%%%%%%%%%%%%%%%%%%%%%%%%%%%%%%%%%%%%%%%%%%%%%%%%%%%%%%%%%%%

\usepackage{color}

\definecolor{darkblue}{rgb}{0,0,.5}
\definecolor{darkgreen}{rgb}{0,.5,0}
\definecolor{darkred}{rgb}{.7,0,0}

%%%%%%%%%%%%%%%%%%%%%%%%%%%%%%%%%%%%%%%%%%%%%%%%%%%%%%%%%%%%%%%%%%%%%%%%%%%%%%%
%                         Font and font like settings                         %
%%%%%%%%%%%%%%%%%%%%%%%%%%%%%%%%%%%%%%%%%%%%%%%%%%%%%%%%%%%%%%%%%%%%%%%%%%%%%%%

\usepackage[charter, greekuppercase=italicized]{mathdesign}
\usepackage{beramono}
\usepackage{berasans}

% Style of vectors and tensors.
\newcommand{\tens}[1]{\boldsymbol{\mathsf{#1}}}
\renewcommand{\vec}[1]{\boldsymbol{#1}}

%%%%%%%%%%%%%%%%%%%%%%%%%%%%%%%%%%%%%%%%%%%%%%%%%%%%%%%%%%%%%%%%%%%%%%%%%%%%%%%
%                               Input encoding                                %
%%%%%%%%%%%%%%%%%%%%%%%%%%%%%%%%%%%%%%%%%%%%%%%%%%%%%%%%%%%%%%%%%%%%%%%%%%%%%%%

\usepackage[T1]{fontenc}
\usepackage[utf8]{inputenc}

%%%%%%%%%%%%%%%%%%%%%%%%%%%%%%%%%%%%%%%%%%%%%%%%%%%%%%%%%%%%%%%%%%%%%%%%%%%%%%%
%                         Hyperrefs and PDF metadata                          %
%%%%%%%%%%%%%%%%%%%%%%%%%%%%%%%%%%%%%%%%%%%%%%%%%%%%%%%%%%%%%%%%%%%%%%%%%%%%%%%

\usepackage{hyperref}
\usepackage{lastpage}

\hypersetup{
	breaklinks=false,
	citecolor=darkgreen,
	colorlinks=true,
	linkcolor=black,
	menucolor=black,
	pdfauthor={Martin Ueding},
	urlcolor=darkblue,
}

%%%%%%%%%%%%%%%%%%%%%%%%%%%%%%%%%%%%%%%%%%%%%%%%%%%%%%%%%%%%%%%%%%%%%%%%%%%%%%%
%                               Math Operators                                %
%%%%%%%%%%%%%%%%%%%%%%%%%%%%%%%%%%%%%%%%%%%%%%%%%%%%%%%%%%%%%%%%%%%%%%%%%%%%%%%

\usepackage[thinspace, squaren]{SIunits}
\usepackage{amsmath}
\usepackage{amsthm}
\usepackage{commath}

% Word like operators.
\DeclareMathOperator{\acosh}{arcosh}
\DeclareMathOperator{\arcosh}{arcosh}
\DeclareMathOperator{\arcsinh}{arsinh}
\DeclareMathOperator{\arsinh}{arsinh}
\DeclareMathOperator{\asinh}{arsinh}
\DeclareMathOperator{\card}{card}
\DeclareMathOperator{\diam}{diam}
\renewcommand{\Im}{\mathop{{}\mathrm{Im}}\nolimits}
\renewcommand{\Re}{\mathop{{}\mathrm{Re}}\nolimits}

% Special single letters.
\DeclareMathOperator{\fourier}{\mathcal{F}}
\newcommand{\C}{\ensuremath{\mathbb C}}
\newcommand{\ee}{\mathrm e}
\newcommand{\ii}{\mathrm i}
\newcommand{\N}{\ensuremath{\mathbb N}}
\newcommand{\R}{\ensuremath{\mathbb R}}
\newcommand{\Z}{\ensuremath{\mathbb Z}}

% Shape like operators.
\DeclareMathOperator{\dalambert}{\Box}
\DeclareMathOperator{\laplace}{\bigtriangleup}
\newcommand{\curl}{\vnabla \times}
\newcommand{\divergence}[1]{\inner{\vnabla}{#1}}
\newcommand{\vnabla}{\vec \nabla}

% Shortcuts
\newcommand{\ev}{\hat{\vec e}}
\newcommand{\e}[1]{\cdot 10^{#1}}
\newcommand{\half}{\frac 12}
\newcommand{\inner}[2]{\left\langle #1, #2 \right\rangle}

% Placeholders.
\newcommand{\emesswert}{\del{\messwert \pm \messwert}}
\newcommand{\fehlt}{\textcolor{darkred}{Hier fehlen noch Inhalte.}}
\newcommand{\messwert}{\textcolor{blue}{\square}}
\newcommand{\punkte}{\textcolor{white}{xxxxx}}

% Separator for equations on a single line.
\newcommand{\eqnsep}{,\quad}

%%%%%%%%%%%%%%%%%%%%%%%%%%%%%%%%%%%%%%%%%%%%%%%%%%%%%%%%%%%%%%%%%%%%%%%%%%%%%%%
%                                  Headings                                   %
%%%%%%%%%%%%%%%%%%%%%%%%%%%%%%%%%%%%%%%%%%%%%%%%%%%%%%%%%%%%%%%%%%%%%%%%%%%%%%%

\usepackage{scrpage2}

\pagestyle{scrheadings}

\automark{section}
\cfoot{\footnotesize{Seite \thepage\ / \pageref{LastPage}}}
\chead{}
\ihead{}
\ohead{\rightmark}
\setheadsepline{.4pt}

%%%%%%%%%%%%%%%%%%%%%%%%%%%%%%%%%%%%%%%%%%%%%%%%%%%%%%%%%%%%%%%%%%%%%%%%%%%%%%%
%                            Bibliography (BibTeX)                            %
%%%%%%%%%%%%%%%%%%%%%%%%%%%%%%%%%%%%%%%%%%%%%%%%%%%%%%%%%%%%%%%%%%%%%%%%%%%%%%%

\newcommand{\bibliographyfile}{../../zentrale_BibTeX/Central}


\hypersetup{
	pdftitle=
}

\ihead{physik321 Spickzettel}
\ifoot{Martin Ueding}
\ofoot{Dieses Werk bzw. Inhalt steht unter einer \href{https://creativecommons.org/licenses/by/3.0/}{CC-BY 3.0} Lizenz}

\title{physik321 Spickzettel}
\author{
	Martin Ueding \\
	\small{\href{mailto:mu@uni-bonn.de}{mu@uni-bonn.de}}
}

\begin{document}

\section{total antisymmetrischer Tensor}

Definition:
\[
	\epsilon_{ijk}
	=
	\begin{cases}
		1 & \text{$ijk$ ist eine gerade Permutation von 1, 2, 3} \\
		-1 & \text{$ijk$ ist eine ungerade Permutation von 1, 2, 3} \\
		0 & \text{sonst}
	\end{cases}
\]

Kontraktion mit einem Index:
\[
	\epsilon_{ijk} \epsilon^{mnk} = \del{
		\delta_i{}^m \delta_j{}^n - \delta_j{}^j \delta_i{}^n
	}
\]

Kontraktion mit drei Indizes:
\[
	\epsilon_{ijk} \epsilon^{ijk} = 3!
\]

\section{$\delta$-Distribution}

definierende Eigenschaft (setze $f \equiv 1$):
\[
	\intop_\Omega \dif x' \, f\del{x'} \delta\del{x - x'}
	=
	\begin{cases}
		f(x) & x \in \Omega \\
		0 & \text{sonst}
	\end{cases}
\]

\section{Elementarladungen und -ströme}

Punktladung:
\[
	\rho\del{\vec x} = \sum_{i = 1}^n q_i \delta\del{\vec x - \vec x_i}
\]

Stromdichte einer Ladungsverteilung:
\[
	\vec j = \rho \vec v
\]

\section{Potentiale}

elektrostatisches Potential:
\[
	\varphi\del{\vec x}
	= \frac{1}{4 \pi \varepsilon_0} \int \dif{V'} \,
	\frac{\rho\del{\vec x'}}{\abs{\vec x - \vec x'}}
\]

magnetostatisches Potential:
\[
	\vec A
	= \frac{\mu_0}{4 \pi} \int \dif{V'} \,
	\frac{\vec j\del{\vec x'}}{\abs{\vec x - \vec x'}}
\]

\section{Felder}

elektrostatische Feldstärke:
\[
	\vec E = - \vnabla \varphi
\]

magnetostatische Induktion:
\[
	\vec B = \curl \vec A
\]

\section{Randwertprobleme}

Green'sche Funktion:
\[
	G\del{\vec x, \vec x'}
	\eqnsep
	\laplace G\del{\vec x, \vec x'}
	= \frac{1}{\varepsilon_0} \delta\del{\vec x - \vec x'} \; \text{innerhalb $V$}
\]

Lösung der Poissongleichung für gegebene Ladungsverteilung $\rho$ (allgemein so
nicht lösbar):
\[
	% TODO Ist die Reihenfolge im hinteren Integral so korrekt?
	\varphi\del{\vec x}
	= \intop_{V} \dif{V'} \, \rho\del{\vec x'} G
	- \varepsilon_0 \ointop_{\partial V} \dif A \,
	\del{\varphi \laplace G - G \laplace \varphi}
\]

\section{Kräfte}

Kraft auf Ladungsdichte:
\[
	\vec F
	= \int \dif{V'} \rho E
\]

Kraft auf Stromfaden und -dichte:
\[
	\vec F_{1,2}
	= I_1 \oint \dif{\vec x_1} \, \times \vec B_2\del{\vec x_1}
	= \int \dif{V} \, \del{\vec j \times \vec B}
\]

Drehmoment $\vec M = \vec x \times \vec F$ auf Stromdichte:
\[
	\vec M
	= \int \dif{V} \, \del{\vec x \times \del{\vec j \times \vec B}}
	\approx \vec m \times \vec B
\]

Ampère'sches Gesetz:
\[
	\vec F_{1,2}
	= \frac{\mu_0}{4 \pi} I_1 I_2 \oint \oint \frac{\dif \vec x_1 \times \del{\dif \vec x_2 \times \vec x_{1,2}}}{x_{1,2}^3}
	= - \frac{\mu_0}{4 \pi} I_1 I_2 \oint \oint \inner{\dif x_1}{\dif x_2} \frac{\vec x_{1,2}}{x_{1,2}^3}
\]

\section{Multipolmomente}

Monopolmoment:
\[
	q
	= \int \dif{V'} \, \rho\del{\vec x'}
\]

Dipolmoment:
\[
	\vec p
	= \int \dif{V'} \, \vec x' \rho\del{\vec x'}
\]

magnetisches Moment:
\[
	\vec m
	= \half \int \dif{V'} \, \del{\vec x \times \vec j}
	= \half I \oint \vec x \times \dif l
\]


%\IfFileExists{\bibliographyfile}{
%	\bibliography{\bibliographyfile}
%	\bibliographystyle{plain}
%}{}

\end{document}

% vim: spell spelllang=de
