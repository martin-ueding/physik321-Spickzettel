\input{header.tex}

\hypersetup{
	pdftitle=
}

\ihead{physik321 Spickzettel}
\ifoot{Martin Ueding}
\ofoot{Dieses Werk bzw. Inhalt steht unter einer \href{https://creativecommons.org/licenses/by/3.0/}{CC-BY 3.0} Lizenz}

\title{physik321 Spickzettel}
\author{
	Martin Ueding \\
	\small{\href{mailto:mu@uni-bonn.de}{mu@uni-bonn.de}}
}

\begin{document}

\section{Vektoridentitäten}

Rotation des Kreuzprodukts:
\[
	\vnabla \times \del{\vec A \times \vec B}
	= \del{\divergence{\vec B} + \inner{\vec B}\vnabla} \vec A
	- \del{\divergence{\vec A} + \inner{\vec A}\vnabla} \vec B
\]


\section{total antisymmetrischer Tensor}

Definition:
\[
	\epsilon_{ijk}
	=
	\begin{cases}
		1 & \text{$ijk$ ist eine gerade Permutation von 1, 2, 3} \\
		-1 & \text{$ijk$ ist eine ungerade Permutation von 1, 2, 3} \\
		0 & \text{sonst}
	\end{cases}
\]

Kontraktion mit einem Index:
\[
	\epsilon_{ijk} \epsilon^{mnk} = \del{
		\delta_i{}^m \delta_j{}^n - \delta_j{}^j \delta_i{}^n
	}
\]

Kontraktion mit drei Indizes:
\[
	\epsilon_{ijk} \epsilon^{ijk} = 3!
\]

\section{$\delta$-Distribution}

definierende Eigenschaft (setze $f \equiv 1$):
\[
	\intop_\Omega \dif x' \, f\del{x'} \delta\del{x - x'}
	=
	\begin{cases}
		f(x) & x \in \Omega \\
		0 & \text{sonst}
	\end{cases}
\]

\section{Elementarladungen und -ströme}

Punktladung:
\[
	\rho\del{\vec x} = \sum_{i = 1}^n q_i \delta\del{\vec x - \vec x_i}
\]

Stromdichte einer Ladungsverteilung:
\[
	\vec j = \rho \vec v
\]

Kontinuitätsgleichung:
\[
	\dod \rho t + \divergence{\vec j} = 0
\]

\section{Potentiale}

elektrostatisches Potential:
\[
	\varphi\del{\vec x}
	= \frac{1}{4 \pi \varepsilon_0} \int \dif{V'} \,
	\frac{\rho\del{\vec x'}}{\abs{\vec x - \vec x'}}
\]

magnetostatisches Potential:
\[
	\vec A
	= \frac{\mu_0}{4 \pi} \int \dif{V'} \,
	\frac{\vec j\del{\vec x'}}{\abs{\vec x - \vec x'}}
\]

\section{Felder}

elektrostatische Feldstärke:
\[
	\vec E = - \vnabla \varphi
\]

magnetostatische Induktion:
\[
	\vec B = \curl \vec A
\]

\[
	\vec B = \frac{\mu_0}{4 \pi} \int \dif{V'} \, \vec j \times \frac{\vec x - \vec x'}{\abs{\vec x - \vec x'}^3}
\]

\section{Randwertprobleme}

Green'sche Funktion:
\[
	G\del{\vec x, \vec x'}
	\eqnsep
	\laplace G\del{\vec x, \vec x'}
	= \frac{1}{\varepsilon_0} \delta\del{\vec x - \vec x'} \; \text{innerhalb $V$}
\]

Lösung der Poissongleichung für gegebene Ladungsverteilung $\rho$ (allgemein so
nicht lösbar):
\[
	% TODO Ist die Reihenfolge im hinteren Integral so korrekt?
	\varphi\del{\vec x}
	= \intop_{V} \dif{V'} \, \rho\del{\vec x'} G
	- \varepsilon_0 \ointop_{\partial V} \dif A \,
	\del{\varphi \laplace G - G \laplace \varphi}
\]

\section{Kräfte}

Kraft auf Ladungsdichte:
\[
	\vec F
	= \int \dif{V'} \rho E
\]

Kraft auf Stromfaden und -dichte:
\[
	\vec F_{1,2}
	= I_1 \oint \dif{\vec x_1} \, \times \vec B_2\del{\vec x_1}
	= \int \dif{V} \, \del{\vec j \times \vec B}
\]

Drehmoment $\vec M = \vec x \times \vec F$ auf Stromdichte:
\[
	\vec M
	= \int \dif{V} \, \del{\vec x \times \del{\vec j \times \vec B}}
	\approx \vec m \times \vec B
\]

Ampère'sches Gesetz:
\[
	\vec F_{1,2}
	= \frac{\mu_0}{4 \pi} I_1 I_2 \oint \oint \frac{\dif \vec x_1 \times \del{\dif \vec x_2 \times \vec x_{1,2}}}{x_{1,2}^3}
	= - \frac{\mu_0}{4 \pi} I_1 I_2 \oint \oint \inner{\dif x_1}{\dif x_2} \frac{\vec x_{1,2}}{x_{1,2}^3}
\]

\section{Multipolmomente}

Monopolmoment:
\[
	q
	= \int \dif{V'} \, \rho\del{\vec x'}
\]

Dipolmoment:
\[
	\vec p
	= \int \dif{V'} \, \vec x' \rho\del{\vec x'}
\]

magnetisches Moment:
\[
	\vec m
	= \half \int \dif{V'} \, \del{\vec x \times \vec j}
	= \half I \oint \vec x \times \dif l
\]

\section{Energie und Leistung}

Leistungsdichte:
\[
	\dod Wt = \int \dif V \inner{\vec j}{\vec E}
\]

Poynting-Vektor:
\[
	\vec S = \vec E \times \vec H
\]

Energiedichte:
\[
	w = \half \del{\inner{\vec D}{\vec E} + \inner{\vec B}{\vec H}}
\]

Poynting'sches Theorem (Kontinuitätsgleichung):
\[
	\dod wt + \divergence{\vec S} + \inner{\vec j}{\vec E} = 0
\]


%\IfFileExists{\bibliographyfile}{
%	\bibliography{\bibliographyfile}
%	\bibliographystyle{plain}
%}{}

\end{document}

% vim: spell spelllang=de
